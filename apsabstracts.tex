\documentclass[11pt]{article}

%\documentclass[11pt]{article}

\usepackage{apsab,graphics,color}

\nofiles


% -------------------------------------------
% Comments
% -------------------------------------------

% Note that the actual abstract submission process must now be done
% by filling out a web form at
% 
%    http://abs.aps.org/
% 
% However, it is recommended to prepare your abstract using this
% template, so that all co-authors can agree on all the details.
% Then, when the time comes to submit the abstract, you will have
% just have to cut and paste the information from this abstract
% into the APS web page.
% 
% Note that it is permissible to use latex constructs in author,
% title, and abstract entries when filling the web form.
%
% For pointers about the process of submitting the abstract to the
% APS web site, see http://www.physics.rutgers.edu/~dhv/aps_abstracts/ .
%
% To use this file you need the APS abstract style file apsab.sty, 
% which should be available with this example file.  
%
% Complete the below information and run LaTeX.  

% The abstract block will be produced inside of a red box.  If the 
% abstract exceeds the red box then you will need to shorten the 
% abstract. Of course this is a rough guide and if you're within
% plus or minus one line of the bottom boarder it is best to check 
% the length using the official APS submission process.  

% This form was created in 2008 when the abstract limit was 1300 
% character.  
%
% Direct inqueries about this example file to Scott Beckman 
% spbeckman@gmail.com
%

% -------------------------------------------
% Header Information (Optional)
% -------------------------------------------

% This information is optional.  It is not included in the 
% abstract block and does not affect the length of the 
% abstract.  

\MeetingID{XXXXX}
%\DateSubmitted{20071126}
%\LogNumber{XXXXX-XXXX-XXXXXX}
\SubmittingMemberSurname{XXXXXXX}
\SubmittingMemberGivenName{XXXXX}
%\SubmittingMemberID{XXX}
\SubmittingMemberEmail{sbeckman@physics.rutgers.edu}
\SubmittingMemberAffil{Rutgers University}
\PresentationType{oral}
\SortCategory{XX.X.X}{}{}{}
\received{XX XXX XXXX}

% -------------------------------------------
% Abstract (Not Optional) 
% -------------------------------------------

\begin{document}

\Title{A theoretical and experimental study of hydrogen storage in
metal organic framework materials.}

% -------------------------------------------
% First Author
% -------------------------------------------

\AuthorSurname{Cooper}
\AuthorGivenName{Valentino R.}
%\AuthorEmail{sbeckman@physics.rutgers.edu}
%\AuthorAffil{Rutgers University}

% Note: Affiliations do NOT need to include the address information.
%       I suggest to keep it short.  If you wish, it could be, e.g.,
%       "Department of Physics and Astronomy, Rutgers University"
%       but the shorter the better.

% Note: In the case of multiple authors with the same affiliation,
%       the affiliation should be left blank except for the last
%       author of the series.  When it comes time to do the actual
%       web submission, if you click "Same as Submitter" to fill
%       out the information for the first author, you might have to
%       erase the Affiliation information if the second author is
%       at the same institution.

% -------------------------------------------
% Subsequent Authors (reproduce to include multiple authors)
% -------------------------------------------

\AuthorSurname{Lee}
\AuthorGivenName{Jeong Yong}

\AuthorSurname{Li}
\AuthorGivenName{Jing}

\AuthorSurname{Chabal}
\AuthorGivenName{Yves}

\AuthorSurname{Langreth}
\AuthorGivenName{David C.}

\AuthorAffil{Rutgers University}

% Select C for Computational or E for Experimental (Optional)
\CategoryType{C}

% 
% Write your abstract here
% 

\begin{abstract}

Metal-organic framework (MOF) materials, assembled by linking metal
ions or clusters through molecular bridges, have been shown to be good
candidates for H$_2$ storage.  We have been successful in fabricating
and characterizing MOFs with increased H$_2$ uptake\footnote{J. Y. Lee
et al.  Adv. Func. Mater., \textbf{17}, 1255 (2007)}, though still too
low for commercial applications.  Here we present a coordinated
theoretical-experimental effort to understand the mechanism of H$_2$
adsorption in true MOF materials. Using the completely \emph{ab
initio} van der Waals density functional (vdW-DF)\footnote{M. Dion et
al. Phys. Rev. Lett., \textbf{92}, 246401
(2004)}$^,$\footnote{T. Thonhauser et al. Phys. Rev. B, \textbf{76},
125112 (2007)} we simulate the interactions of H$_2$ within
Zn$_2$(bdc)$_2$(ted).  We demonstrate that modeling the entire MOF
structure can result in different H$_2$ adsorption geometries, binding
energies and vibrational frequencies than observed in calculations on
fragments of the MOF.  Combining these results with experimental IR
vibrational frequency studies may provide insights into modifying MOF
structure and composition for enhanced H$_2$ uptake.

\end{abstract}


% -------------------------------------------
% The Red Box -- Do Not Edit
% -------------------------------------------

% Yes, this is a crude box, but it works....

% top line
\vspace{-5.64125in}
\hspace{0.8125in}
\color{red}\rule{4.7812in}{0.02in}\color{black}

% bottom line
\vspace{4.1875in}
\hspace{0.8125in}
%\color{red}\rule[0.1in]{4.75in}{0.02in}\color{black}
\color{red}\rule{4.7812in}{0.02in}\color{black}

%left line
\vspace{-4.4062in}
\hspace{0.81250in}
\color{red}\rule{0.02in}{4.3750in}\color{black}

%right line
\vspace{-4.3750in}
\hspace{5.5738in}
\color{red}\rule{0.02in}{4.3750in}\color{black}


\end{document}